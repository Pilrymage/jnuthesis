% -*- coding: utf-8 -*-
% !TEX program = xelatex
\documentclass{jnuthesis}

\begin{document}

\renewcommand{\title}{A thesis class for Jinan University} % 英文标题
\renewcommand{\biaoti}{暨南大学课程论文模板}  % 中文标题
\renewcommand{\mingcheng}{《课程论文写作》}
\renewcommand{\leibie}{通识教育必修课}
\renewcommand{\xingming}{暨小小}
\renewcommand{\xuehao}{2021123456}
\renewcommand{\xueyuan}{信息科学技术学院}
\renewcommand{\xuexi}{计算机科学系}
\renewcommand{\zhuanye}{网络空间安全}
\renewcommand{\jiaoshi}{暨大大}
\renewcommand{\danwei}{暨大大}

% \titlepage % 论文封面

%\statement % 诚信声明

\begin{zhabstract}
\zhaiyao
摘要应扼要叙述本论文的主要内容、特点,文字要精炼,是一篇具有独立性和完整性的短文,应包括本论文的主要成果和结论性意见。
摘要中不宜使用公式、图表,不标注引用文献编号,避免将摘要写成目录式的内容介绍,也不要将摘要写成“前言”。
编写摘要应注意:客观反映原文内容,不得简单地重复题名中已有的信息,要着重反映论文的新内容和特别强调的观点。
摘要宜采用第三人称过去式的写法(如“对……进行了研究”,“综述了……”等;不应写成“本文”、“我校……”等)。
摘要不分段,以300--400字左右为宜。可在写完初稿时再写摘要。
\guanjianci
关键词1;关键词2;关键词3;关键词4;关键词5
\end{zhabstract}

\chapter{绪论}

\section{文献综述}

/*介绍该研究的国内外现状,已取得的成果等。可分小节介绍。*/

\section{研究框架}

/*阐述本论文的研究目标、研究内容、创新之处、研究方法等。*/

\section{术语说明}

/*解释本文中出现的术语。*/

\chapter{系统分析}

\section{节}

\subsection{小节}

\subsubsection{小小节}

%\show\thesubsubsection

\chapter{系统设计}

\appendix

\chapter*{结论}\addcontentsline{toc}{chapter}{\hspace{-1em}结论}

/*结论作为单独一章排列,但不加章号。
结论是对整个论文主要成果的归纳,要突出设计(论文)的创新点,
以简练的文字对论文的主要工作进行评价,一般为400~1 000字。*/

\chapter*{致谢}\addcontentsline{toc}{chapter}{\hspace{-1em}致谢}

感谢我的导师XXX老师,谢谢他对我的悉心指导。
他无私的关爱和严谨的治学态度,将激励我不断的进取,走好以后的道路。
其次,还要感谢在这四年的学习中教过我的所有老师们,谢谢他们传授给了我知识。
我的同学XXX,在写作的过程中给我提供了一些宝贵的资料和建议,在此一并感谢!

\chapter*{附录A}\addcontentsline{toc}{chapter}{\hspace{-1em}附录A}

/*是正文主体的补充项目,并不是必需的。下列内容可以作为附录:
(1)为了整篇材料的完整,插入正文又有损于编排条理性和逻辑性的材料;
(2)由于篇幅过大,或取材于复制件不便编入正文的材料;
(3)对一般读者并非必须阅读,但对本专业人员有参考价值的资料;
(如外文文献复印件及中文译文、公式的推导、程序流程图、图纸、数据表格等)
附录按“附录A,附录B,附录A1“等编号。
请单击样式“附录1”为第1级的附录编号,样式“附录2”为第二级的附录编号,样式“附录3”控制第三级别的样式。*/

\chapter*{参考文献}\addcontentsline{toc}{chapter}{\hspace{-1em}参考文献}

引用文献标示应置于所引内容最末句的右上角。所引文献编号用阿拉伯数字置于方括号“[ ]”中,如“二次铣削[1]”。
如同一处引用了多个文献,文献编号间用逗号分隔,如“二次铣削[1,3] ”。
当提及的参考文献为文中直接说明时,其序号应该与正文排齐,如“由文献[8,10~14]可知”。
经济、管理类论文引用文献,若引用的是原话,要加引号,一般写在段中;若引的不是原文只是原意,
文前只需用冒号或逗号,而不用引号。在参考文献之外,若有注释的话,建议采用夹注,即紧接文句,用圆括号标明。
或者以脚注的形式排在页面底端,按①,②,③编号。

\end{document}
